\section{Injeção de Dependências}

O \textit{pattern} de injeção de dependências é uma técnica que pretende diminuir as junções entre classes tornando assim a evolução do \textit{software} mais suave e fácil. É uma das formas de se fazer Inversão de Controlo, descrita na secção anterior.

Existem três formas principais de se fazer injeção de dependências, através de \textit{Constructor Injection}, \textit{Setter Injection} e \textit{Interface Injection} ou por um \textit{Service Locator}.

Nas próximas subseções iremos abordar estes métodos, apresentando algumas soluções.

\subsection{Constructor Injection}
Neste método, as dependências do objeto são injetadas diretamente no construtor. Depois da classe \textit{Stand} ser instanciada é possível atribuir à variável de instância \textit{veiculo} o objeto do qual ela depende.

\begin{lstlisting}[caption=Injeção pelo Construtor]
public class Stand {
  private Veiculo veiculo;

  public Stand(Veiculo veiculo) {
    this.veiculo = veiculo;
  }
}
\end{lstlisting}

\subsection{Setter Injection}
Neste procedimento é utlizado o método \textit{setter}, \textit{setVeiculo}, que permite injectar através dos argumentos de entrada do método as dependências para a variável de instância do objeto.

\begin{lstlisting}[caption=Injeção pelo Setter]
public class Stand {
  private Veiculo veiculo;

  @Inject
  @Required
  public void setVeiculo(Veiculo veiculo) {
    this.veiculo = veiculo;
  }
}
\end{lstlisting}

\subsection{Interface Injection}
Desenhou-se uma interface onde os métodos \textit{setters} aceitam dependências. Então quando se implementa esta interface na classe \textit{Stand} é necessário definir os métodos da interface e através do método \textit{setVeiculo} é possível fazer injeção de dependências dos objetos da classe.

\begin{lstlisting}[caption=Injeção pela Interface]
public interface injectVeiculo {
  public void setVeiculo(Veiculo veiculo);
}

public class Stand implements injectVeiculo {
  private Veiculo veiculo;

  public void setVeiculo(Veiculo veiculo) {
    this.veiculo = veiculo;
  }
}
\end{lstlisting}

\subsection{Service Locator}

\begin{lstlisting}[caption=Injeção por Service Locator]
public class ServiceLocator {
  public static Stand stand() {
    return
  }
}


public class Stand {
  private Veiculo veiculo;

}
\end{lstlisting}


class MovieLister...
  MovieFinder finder = ServiceLocator.movieFinder();

class ServiceLocator...
  public static MovieFinder movieFinder() {
      return soleInstance.movieFinder;
  }
  private static ServiceLocator soleInstance;
  private MovieFinder movieFinder;
