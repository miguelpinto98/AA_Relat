\section{Conclusão}

Como é sabido, o conhecimento e a aplicação de técnicas e modelos de programação no desenvolvimento de uma aplicação, designados vulgarmente por \textit{design patterns}, influirá, de forma bastante positiva, a qualidade do código desenvolvido tanto a nível organizacional como a nível estrutural.
Assim, o desacopolamento do código é substancialmente maior, fazendo com que cada classe (referindo o mundo de \textit{O.O.P.}) defina completamente a sua estrutra e comportamento e que a definição desta não dependa de terceiras classes, reduzindo, assim, o número de dependências entre estas.

No entanto, a adoção destes padrões de desenvolvimento requer um tempo extra de desenvolvimento que, para projetos de média ou pequena dimensão, pode não ser necessário.
A complexidade da solução desenvolvida para o problema em questão corre o risco, também, de aumentar substancialmente e, assim, provocar um \textit{overhead} no que se refere ao tempo de execução da aplicação.

Em suma, é aconselhado que tais padrões e técnicas sejam adotadas no desenvolvimento da aplicação, aumentando o tempo de desenvolvimento mas produzindo-se um código de maior qualidade e mais apto a ser testado e mantido no futuro.
