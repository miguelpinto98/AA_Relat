\documentclass[twoside]{article}

\usepackage[utf8]{inputenc}
\usepackage[portuges]{babel}

\usepackage{lipsum} % Package to generate dummy text throughout this template

\usepackage[sc]{mathpazo} % Use the Palatino font
\usepackage[T1]{fontenc} % Use 8-bit encoding that has 256 glyphs
\linespread{1.05} % Line spacing - Palatino needs more space between lines
\usepackage{microtype} % Slightly tweak font spacing for aesthetics

\usepackage[hmarginratio=1:1,top=32mm,columnsep=20pt]{geometry} % Document margins
\usepackage{multicol} % Used for the two-column layout of the document
\usepackage[hang, small,labelfont=bf,up,textfont=it,up]{caption} % Custom captions under/above floats in tables or figures
\usepackage{booktabs} % Horizontal rules in tables
\usepackage{float} % Required for tables and figures in the multi-column environment - they need to be placed in specific locations with the [H] (e.g. \begin{table}[H])
\usepackage{hyperref} % For hyperlinks in the PDF

\usepackage{lettrine} % The lettrine is the first enlarged letter at the beginning of the text
\usepackage{paralist} % Used for the compactitem environment which makes bullet points with less space between them

\usepackage{abstract} % Allows abstract customization
\renewcommand{\abstractnamefont}{\normalfont\bfseries} % Set the "Abstract" text to bold
\renewcommand{\abstracttextfont}{\normalfont\small\itshape} % Set the abstract itself to small italic text

\usepackage{titlesec} % Allows customization of titles
\renewcommand\thesection{\Roman{section}} % Roman numerals for the sections
\renewcommand\thesubsection{\Roman{subsection}} % Roman numerals for subsections
\titleformat{\section}[block]{\large\scshape\centering}{\thesection.}{1em}{} % Change the look of the section titles
\titleformat{\subsection}[block]{\large}{\thesubsection.}{1em}{} % Change the look of the section titles

\usepackage{fancyhdr} % Headers and footers
\pagestyle{fancy} % All pages have headers and footers
\fancyhead{} % Blank out the default header
\fancyfoot{} % Blank out the default footer
\fancyhead[C]{Engenharia de Aplicações $\bullet$ Arquiteturas Aplicacionais $\bullet$ {\today}} % Custom header text
\fancyfoot[RO,LE]{\thepage} % Custom footer text

%----------------------------------------------------------------------------------------

\usepackage{listings}
\usepackage{color}

\definecolor{dkgreen}{rgb}{0,0.6,0}
\definecolor{gray}{rgb}{0.5,0.5,0.5}
\definecolor{mauve}{rgb}{0.58,0,0.82}

\lstset{frame=tb,
  language=Java,
  aboveskip=3mm,
  belowskip=3mm,
  showstringspaces=false,
  columns=flexible,
  basicstyle={\small\ttfamily},
  numbers=none,
  numberstyle=\tiny\color{gray},
  keywordstyle=\color{blue},
  commentstyle=\color{dkgreen},
  stringstyle=\color{mauve},
  breaklines=true,
  breakatwhitespace=true,
  tabsize=3
}

\renewcommand\lstlistingname{Exemplo}

%----------------------------------------------------------------------------------------
%	TITLE SECTION
%----------------------------------------------------------------------------------------

\title{\vspace{-10mm}\fontsize{22pt}{10pt}\selectfont\textbf{Inversão de Controlo \& Injeção de Dependências}} % Article title

\author{
  \large
  \textsc{José Morgado}\\
  \normalsize \href{mailto:pg27759@alunos.uminho.pt}{pg27759@alunos.uminho.pt} % Your email address
  \vspace{-5mm}
\and
  \large
  \textsc{Luís Miguel Pinto}\\
  \normalsize \href{mailto:pg27756@alunos.uminho.pt}{pg27756@alunos.uminho.pt}\\ % Your email address
  \vspace{-5mm}
\and
  \large
  \textsc{Pedro Carneiro}\\
  \normalsize \href{mailto:pg25324@alunos.uminho.pt}{pg25324@alunos.uminho.pt} % Your email address
  \vspace{-5mm}
}

\date{}
%----------------------------------------------------------------------------------------

\begin{document}

\maketitle % Insert title

\thispagestyle{fancy} % All pages have headers and footers

%	ABSTRACT
%----------------------------------------------------------------------------------------
\begin{abstract}

\noindent \lipsum[1] % Dummy abstract text

\end{abstract}


%	ARTICLE CONTENTS
%----------------------------------------------------------------------------------------
\begin{multicols}{2} % Two-column layout throughout the main article text

\section{Introdução}

Tem-se assistido a um crescimento da complexidade das arquiteturas dos sistemas de software, sendo cada vez mais comum a integração de múltiplos componentes. Hoje em dia, ninguém desenvolve aplicações web sem recorrer a frameworks e web services.

De modo a que isto seja possível, precisamos de técnicas que permitam garantir a modularidade e extensibilidade dos componentes de software. Foi neste contexto que surgiram os conceitos de Inversão de Controlo e Injeção de Dependências.


\section{Inversão de controlo}

A Inversão de Controlo é um princípio de concepção de software. Tipicamente, quando queremos construir um programa apenas codificamos parte do mesmo e recorremos a biblitecas para realizar parte das tarefas; neste caso, o nosso código (específico) depende de código externo (reutilizável). Usando Inversão de Controlo, a abordagem é a oposta: o código reutilizável das bibliotecas é que vai necessitar/depender do nosso código específico.

Um exemplo disso mesmo são as frameworks. Estas coordenam a execução das aplicações, o que nos permite ter que definir apenas parte do comportamento das mesmas. Ou seja, funcionam como esqueletos aos quais apenas temos de disponibilizar os plugins necessários.

Por forma a melhor explicar este princípio, atente-se no extrato de código retirado e adaptado de um \textit{post} de \textit{Matin Fowler} onde aborda, precisamente, esta questão:

\begin{lstlisting}[caption=]
puts 'What is your name?'
name = gets
process_name(name)
\end{lstlisting}

Como se pode claramente verificar, o controlo de fluxo deste pequeno programa é inteiramente controlado pelo programador.
Ele tem a liberdade total de decidir quando perguntar, de quando ler a resposta e de quando a processar.

No entanto, e para se ter uma ideia prática do conceito de Inversão de Controlo, imagine-se, agora, que em vez de se efetuar a leitura através da consola, faz-se recorrendo a um ambiente gráfico:

\begin{lstlisting}[caption=]
require 'tk'
root = TkRoot.new()
name_label = TkLabel.new() {text "What is Your Name?"}
name_label.pack
name = TkEntry.new(root).pack
name.bind("FocusOut") {process_name(name)}
Tk.mainloop()
\end{lstlisting}

Como se pode constatar, acontece que agora o programador já não tem controlo da decisão de invocar o método que processará o nome introduzido.
Assim, cabe à biblioteca de ambiente gráfico decidir quando deverá fazer a invocação do método baseado no \textit{binding} feito.
Portanto, o controlo foi invertido: em vez de ser o programador a invocar o código da biblioteca, é a biblioteca que invoca o código da aplicação.
Este é o conceito que está na base da Inversão do Controlo.

Há várias técnicas que seguem este princípio, como a Injeção de Depedências, que será apresentada na secção seguinte.


\section{Injeção de Dependências}

O \textit{pattern} de injeção de dependências é uma técnica que pretende diminuir as junções entre classes tornando assim a evolução do \textit{software} mais suave e fácil. É uma das formas de se fazer Inversão de Controlo, descrita na secção anterior.

Existem três formas principais de se fazer injeção de dependências, através de \textit{Constructor Injection}, \textit{Setter Injection} e \textit{Interface Injection} ou por um \textit{Service Locator}.

Nas próximas subseções iremos abordar estes métodos, apresentando algumas soluções.

\subsection{Constructor Injection}
Neste método, as dependências do objeto são injetadas diretamente no construtor. Depois da classe \textit{Stand} ser instanciada é possível atribuir à variável de instância \textit{veiculo} o objeto do qual ela depende.

\begin{lstlisting}[caption=Injeção pelo Construtor]
public class Stand {
  private Veiculo veiculo;

  public Stand(Veiculo veiculo) {
    this.veiculo = veiculo;
  }
}
\end{lstlisting}

\subsection{Setter Injection}
Neste procedimento é utlizado o método \textit{setter}, \textit{setVeiculo}, que permite injectar através dos argumentos de entrada do método as dependências para a variável de instância do objeto.

\begin{lstlisting}[caption=Injeção pelo Setter]
public class Stand {
  private Veiculo veiculo;

  @Inject
  @Required
  public void setVeiculo(Veiculo veiculo) {
    this.veiculo = veiculo;
  }
}
\end{lstlisting}

\subsection{Interface Injection}
Desenhou-se uma interface onde os métodos \textit{setters} aceitam dependências. Então quando se implementa esta interface na classe \textit{Stand} é necessário definir os métodos da interface e através do método \textit{setVeiculo} é possível fazer injeção de dependências dos objetos da classe.

\begin{lstlisting}[caption=Injeção pela Interface]
public interface injectVeiculo {
  public void setVeiculo(Veiculo veiculo);
}

public class Stand implements injectVeiculo {
  private Veiculo veiculo;

  public void setVeiculo(Veiculo veiculo) {
    this.veiculo = veiculo;
  }
}
\end{lstlisting}

\subsection{Service Locator}

\begin{lstlisting}[caption=Injeção por Service Locator]
public class ServiceLocator {
  public static Stand stand() {
    return
  }
}


public class Stand {
  private Veiculo veiculo;

}
\end{lstlisting}


class MovieLister...
  MovieFinder finder = ServiceLocator.movieFinder();

class ServiceLocator...
  public static MovieFinder movieFinder() {
      return soleInstance.movieFinder;
  }
  private static ServiceLocator soleInstance;
  private MovieFinder movieFinder;



%	REFERENCE LIST
%----------------------------------------------------------------------------------------
\section{Referências}

\begin{thebibliography}{99} % Bibliography - this is intentionally simple in this template

\bibitem{Freeman}
Eric Freeman, Elisabeth Robson, Bert Bates, Kathy Sierra. Head First Design Patterns. O'Reilly Media, 2004.

\bibitem{wiki}
http://wiki.portugal-a-programar.pt/dev\_geral:java:padrao\_singleton

\bibitem{devmedia}
http://www.devmedia.com.br/padrao-de-projeto-singleton-em-java/26392

\bibitem{tutorialspoint}
http://www.tutorialspoint.com/design\_pattern/facade\_pattern.htm

\bibitem{Gamma}
Design Patterns: Elements of Reusable Object-Oriented Software, Erich Gamma, Richard Helm, Ralph Johnson, John M. Vlissides, November 1994

\end{thebibliography}


\end{multicols}

\end{document}
