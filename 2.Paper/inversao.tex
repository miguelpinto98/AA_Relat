\section{Inversão de controlo}

A Inversão de Controlo é um princípio de concepção de software. Tipicamente, quando queremos construir um programa apenas codificamos parte do mesmo e recorremos a biblitecas para realizar parte das tarefas; neste caso, o nosso código (específico) depende de código externo (reutilizável). Usando Inversão de Controlo, a abordagem é a oposta: o código reutilizável das bibliotecas é que vai necessitar/depender do nosso código específico.

Um exemplo disso mesmo são as frameworks. Estas coordenam a execução das aplicações, o que nos permite ter que definir apenas parte do comportamento das mesmas. Ou seja, funcionam como esqueletos aos quais apenas temos de disponibilizar os plugins necessários.

Há várias técnicas que seguem este princípio, como a Injeção de Depedências, que será apresentada na secção seguinte.
