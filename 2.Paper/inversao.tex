\section{Inversão de controlo}

A Inversão de Controlo é um princípio de concepção de software. Tipicamente, quando queremos construir um programa apenas codificamos parte do mesmo e recorremos a biblitecas para realizar parte das tarefas; neste caso, o nosso código (específico) depende de código externo (reutilizável). Usando Inversão de Controlo, a abordagem é a oposta: o código reutilizável das bibliotecas é que vai necessitar/depender do nosso código específico.

Um exemplo disso mesmo são as frameworks. Estas coordenam a execução das aplicações, o que nos permite ter que definir apenas parte do comportamento das mesmas. Ou seja, funcionam como esqueletos aos quais apenas temos de disponibilizar os plugins necessários.

Por forma a melhor explicar este princípio, atente-se no extrato de código retirado e adaptado de um \textit{post} de \textit{Matin Fowler} onde aborda, precisamente, esta questão:

\begin{lstlisting}[caption=]
puts 'What is your name?'
name = gets
process_name(name)
\end{lstlisting}

Como se pode claramente verificar, o controlo de fluxo deste pequeno programa é inteiramente controlado pelo programador.
Ele tem a liberdade total de decidir quando perguntar, de quando ler a resposta e de quando a processar.

No entanto, e para se ter uma ideia prática do conceito de Inversão de Controlo, imagine-se, agora, que em vez de se efetuar a leitura através da consola, faz-se recorrendo a um ambiente gráfico:

\begin{lstlisting}[caption=]
require 'tk'
root = TkRoot.new()
name_label = TkLabel.new() {text "What is Your Name?"}
name_label.pack
name = TkEntry.new(root).pack
name.bind("FocusOut") {process_name(name)}
Tk.mainloop()
\end{lstlisting}

Como se pode constatar, acontece que agora o programador já não tem controlo da decisão de invocar o método que processará o nome introduzido.
Assim, cabe à biblioteca de ambiente gráfico decidir quando deverá fazer a invocação do método baseado no \textit{binding} feito.
Portanto, o controlo foi invertido: em vez de ser o programador a invocar o código da biblioteca, é a biblioteca que invoca o código da aplicação.
Este é o conceito que está na base da Inversão do Controlo.

Há várias técnicas que seguem este princípio, como a Injeção de Depedências, que será apresentada na secção seguinte.
