\subsection{Facade}

O \textit{pattern} Facade esconde toda a complexidade de uma ou mais classes do sistema e fornece uma interface, que o cliente pode aceder  para utilizar o sistema.

Ao se utilizar uma Facade, que implementa uma interface, pretende-se reduzir o nível de complexidade no subsistema, aumentando a facilidade de utilização.\\

\begin{figure}[!h]
\centering
\includegraphics[scale=0.5]{img/facade-estrutura}
\caption{Antes e depois da utilização do Facade}
\end{figure}

Como vemos na figura anterior, este \textit{pattern} é definido através de uma única classe que fornece os métodos necessários ao cliente e envia os pedidos deste às classes do subsistema.\\

\textbf{Aplicação}

Utilizar Facade quando:

\begin{itemize}
  \item Se deseja fornecer uma interface simples para um sistema complexo.
  \item Existe muitas dependências entre clientes e as classes de implementação de uma abstração.
  \item Se pretende ter uma estrutura em camadas no sistema.\\
\end{itemize}

\textbf{Colaboração}

Os clientes comunicam com o subsistema através do envio de pedidos ao Facade, que redireciona os mesmos para as classes apropriadas do subsistema. Ainda que, os clientes que utilizam este \textit{pattern} não tem que aceder aos objetos do seu subsistema diretamente.\\

\textbf{Vantagens}

\begin{itemize}
\item A utilização deste \textit{pattern} permite diminuir as ligações entre os clientes e o sistema.
\item Para adicionar novas funcionalidades ao sistema seria necessário alterar apenas o Facade, ao invés de alterar os vários objetos do sistema, sem afetar o cliente.\\
\end{itemize}


\textbf{Implementação}

\begin{figure}[!h]
\centering
\includegraphics[scale=0.5]{img/facade_diagrama}
\caption{Antes e depois da utilização do Facade}
\end{figure}

% implementação
 Nós vamos criar uma interface Forma e classes concretas implementando a interface Shape. A ShapeMaker classe de fachada é definido como um próximo passo.

% Classe ShapeMaker usa as classes concretas para delegar as chamadas de usuários para essas classes. FacadePatternDemo, a nossa classe demo, usará classe ShapeMaker para mostrar os resultados.

