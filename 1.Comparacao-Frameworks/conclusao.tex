\section{Conclusão}

No presente relatório foram analisadas algumas das várias frameworks em Java presentes em cada uma das três camadas, tentado ter um suporte para a escolha da melhor quando for necessário. Para a escolha de uma framework é necessário analisar o contexto em que esta irá estar inserida, e o tipo de projeto que se pretende construir.

A separação de camadas permite tornar os sistemas mais flexivéis, de modo a que se altere cada camada sem necessidade de se alterarem as outras. As camadas analisadas anteriormente são as principais, mas é ainda possível dividir estas camadas em subcamadas.

Com a elaboração deste trabalho, conseguímos perceber que não se deve optar por uma framework por ser uma tendência do momento, mas sim refletir sobre qual, para o projeto pretendido irá tirar mais partidos das características descritas ao longo do relatório dentro da própria infraestrutura computacional.
