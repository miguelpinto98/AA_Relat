\section{Introdução}

O desenvolvimento de aplicações e as metodologias aplicadas a este processo sempre foram alvo de estudo cuidado e intensivo por parte da comunidade de engenheiros de \textit{software}.

Um dos modelos deste paradigma que mais aceitação reuniu na comunidade foi o modelo de três camadas (\textit{Three Tier Architecture}).
Este defende a separação clara no desenvolvimento da aplicação de conceitos distintos: a \textbf{interface com o utilizador}, a \textbf{lógica de negócio} e os \textbf{dados}.
Assim, neste modelo, cada um destes conceitos terá mapeamento direto na sua camada respetiva:

\begin{description}

\item[Camada de Apresentação]
É a camada responsável pelo desenvolvimento da interação entre o sistema e o utilizador.
É nesta camada que os pedidos dos clientes serão inicialmente processados e, também, será nesta onde serão exibidos os resultados a estes dos seus pedidos.

\item[Camada de Negócio]
É nesta camada que se encontra a solução desenhada para o problema.
Esta apresenta propriedades de dados voláteis.
Assim, por forma a persisti-los, terá de comunicar com a camada de dados.

\item[Camada de Dados]
Esta camada é responsável por armazenar os dados do problema.
Não pode, por isso, ser desassociada a uma base de dados.

\end{description}

O objetivo deste trabalho passa por identificar e analisar, para cada uma destas camadas, primeiro individual e depois globalmente, \textit{frameworks} no âmbito da linguagem de programação \textit{Java} que possam facilitar o processo de desenvolvimento de \textit{software}.\\
Assim, para cada camada, e mediante a opinião do grupo, será indicada a \textit{framework} que melhor serve os interesses do programador.
